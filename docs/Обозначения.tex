\section*{ОБОЗНАЧЕНИЯ И СОКРАЩЕНИЯ}

Графический движок -- приложение или компонент приложения, отвечающий за обработку виртуальных графических объектов и вывод их изображения на экран.

Рендеринг -- процесс обработки двух- или трёхмерного объекта и вывода его проекции на экран.

Спрайты -- двухмерные графические объекты.

Ray casting -- <<бросание лучей>>, простейший метод определения видимости объектов в 3D-сцене.

Ray tracing -- трассировка лучей, метод определения видимости объектов в 3D-сцене, основанный на физических закономерностях светового распространения.

API (Application Programming Interface) -- интерфейс программирования приложений.

Кроссплатформенность -- возможность использования одного и того же приложения на разных платформах (операционных системах).

Скриптинг -- задание логики и поведения объектов в сцене.

Сцена -- абстрактное представление совокупности объектов, взаимодействующих между собой и обрабатываемых движком одновременно.

VBO (Vertex Buffer Object) -- объект буфера вершин.

VAO (Vertex Array Object) -- объект массива вершин.

EBO (Element Buffer Object) -- объект буфера элементов.

ИС -- информационная система.

ИТ -- информационные технологии.

КТС -- комплекс технических средств.

ПО -- программное обеспечение.

ОС -- операционная система.

РП -- рабочий проект.

ТЗ -- техническое задание.

ТП -- технический проект.

UML (Unified Modelling Language) -- язык графического описания для объектного моделирования в области разработки программного обеспечения.
