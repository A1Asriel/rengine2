\section{Анализ предметной области}
\subsection{Характеристика предприятия и его деятельности}

Компания ОАО <<Какао-Какао>> является ведущим интегратором корпоративных информационных систем, специализируясь на внедрении и сопровождении решений на платформе 1С, а также системном администрировании сложных ИТ-инфраструктур. Основанная в 1996 году, компания активно развивает направление автоматизации бизнес-процессов и создания специализированного программного обеспечения для своих клиентов.

В современном мире компьютерная графика играет важнейшую роль в различных областях: от игровой индустрии до научных исследований. Технологии рендеринга трёхмерных объектов прошли долгий путь развития от простых каркасных моделей до фотореалистичных изображений с динамическим освещением. Важнейшим этапом в этой эволюции стало появление графических API, таких как OpenGL и DirectX, которые позволили разработчикам создавать высокопроизводительные приложения для визуализации 3D-графики.

Термин <<графический движок>> (англ. rendering engine) обозначает программный комплекс, предназначенный для визуализации виртуальных сцен. Современные графические движки предоставляют широкий спектр возможностей: от базовой отрисовки геометрических примитивов до сложных систем освещения, текстурирования и физического моделирования. Они являются фундаментальным инструментом в разработке интерактивных приложений и игр.

Графические движки находят применение во множестве областей: разработка компьютерных игр, архитектурная визуализация, кинематограф, научная визуализация данных, системы виртуальной и дополненной реальности, тренажеры и симуляторы. Каждая из этих областей предъявляет свои требования к функциональности и производительности движков.

В рамках преддипломной практики под руководством ОАО <<Какао-Какао>> был создан графический движок REngine, предназначенный для интеграции с корпоративными системами визуализации данных. Движок использует технологии OpenGL и SDL2, обеспечивающие кроссплатформенность и возможность встраивания в существующие ИТ-инфраструктуры клиентов. Данное решение может быть использовано для создания специализированных интерфейсов визуализации производственных процессов, интеграции с системами бизнес-аналитики и другими корпоративными решениями.

\subsection{История развития графических движков}

Эволюция графических движков тесно связана с развитием аппаратного обеспечения и графических API. Можно выделить несколько ключевых этапов:

\begin{itemize}
\item 1980-1989 гг. -- появление первых примитивных движков для аркадных автоматов и домашних компьютеров. Графика была исключительно двумерной, использовались спрайты и плиточные карты.
\item 1990-1999 гг. -- с появлением первых 3D-ускорителей начинается эра трёхмерной графики. Знаковые движки этого периода:
    \begin{itemize}[itemindent=\parindent,leftmargin=\parindent]
    \item Wolfenstein 3D engine (1992) -- первый движок с псевдо-3D графикой, использующий технологию ray casting;
    \item Doom engine (1993) -- развитие предыдущего движка, дополняющее его вертикальной видимостью;
    \item Quake engine (1996) -- первый полностью трёхмерный движок с аппаратным ускорением;
    \item Unreal Engine (1998) -- ввел концепции визуального редактора сцен и скриптинга.
    \end{itemize}
\item 2000-2009 гг. -- стандартизация графических API (OpenGL, DirectX), появление шейдеров и программируемого конвейера, развитие middleware -- программ-посредников (OGRE, Irrlicht);
\item 2010-2019 гг. -- доминирование коммерческих движков (Unity, Unreal Engine 4), развитие мобильной графики (OpenGL ES), появление новых API (Vulkan, Metal);
\item 2020-н.в. -- акцент на фотореализм с использованием аппаратно ускоренной трассировки лучей (ray tracing), виртуальную реальность (VR) и кроссплатформенность, растет спрос на специализированные движки для бизнес-приложений и визуализации данных.
\end{itemize}

В корпоративном секторе графические движки находят применение в:
\begin{itemize}
\item системах визуализации производственных процессов;
\item инструментах бизнес-аналитики и BI-системах;
\item тренажерах и симуляторах для промышленности;
\item системах проектирования и прототипирования.
\end{itemize}

\subsection{Графические движки и технологии рендеринга}

Основное преимущество использования графических движков заключается в абстрагировании от низкоуровневых деталей работы с графическим API. Разработчик получает готовый инструментарий для создания и визуализации трёхмерных сцен, что существенно сокращает время разработки и позволяет сосредоточиться на логике приложения.

Современные графические движки можно классифицировать по нескольким критериям.

По используемому графическому API:
\begin{itemize}
\item OpenGL -- обеспечивает совместимость с наиболее широким количеством платформ;
\item DirectX -- оптимизирован для работы на машинах под управлением Microsoft Windows;
\item Vulkan -- является наследником OpenGL и обеспечивает низкоуровневый доступ к графическому оборудованию для наибольшей производительности;
\item Metal -- разработан специально для устройств Apple.
\end{itemize}

По типу рендеринга:
\begin{itemize}
\item движки с отложенным рендерингом (Deferred Rendering) -- позволяют эффективно обрабатывать сложные графические сцены с большим количеством источников света;
\item движки с прямым рендерингом (Forward Rendering) -- обеспечивают высокую производительность при упрощённых графических сценах;
\item гибридные решения, сочетающие преимущества обоих подходов.
\end{itemize}

По специализации:
\begin{itemize}
\item универсальные движки -- подходят для широкого спектра задач;
\item специализированные движки -- оптимизированы для конкретных типов приложений (игры, научная визуализация и т.д.);
\item движки реального времени -- обеспечивают мгновенную визуализацию изменений в сцене.
\end{itemize}

Движок REngine относится к категории универсальных OpenGL-движков с прямым рендерингом, что обеспечивает оптимальный баланс между производительностью и качеством визуализации для широкого спектра приложений.
