\section*{ВВЕДЕНИЕ}
\addcontentsline{toc}{section}{ВВЕДЕНИЕ}

Современные технологии визуализации трёхмерной графики нашли широкое применение в различных областях: от компьютерных игр и кинематографа до научной визуализации и проектирования. Одним из ключевых компонентов, обеспечивающих работу с 3D-графикой, являются графические движки — специализированные программные комплексы, отвечающие за рендеринг трёхмерных сцен в реальном времени.

Графический движок представляет собой сложную систему, включающую в себя подсистемы для работы с геометрией, материалами, текстурами, освещением, анимацией и другими аспектами компьютерной графики. Современные движки должны обеспечивать высокую производительность, кроссплатформенность и гибкость при работе с различными типами контента.

В последние годы наблюдается растущий спрос на специализированные решения для визуализации, которые могут быть интегрированы в существующие информационные системы предприятий. Это требует от разработчиков создания модульных, расширяемых и хорошо документированных решений, способных работать в различных окружениях.

Разрабатываемый в рамках данной работы графический движок ставит своей целью предоставить простое в использовании, но мощное решение для визуализации трёхмерных сцен. Движок построен на базе современных технологий, обеспечивающих кроссплатформенность и высокую производительность.

\emph{Целью настоящей работы} является разработка кроссплатформенного графического движка для визуализации трёхмерных сцен с использованием современных технологий рендеринга. Для достижения поставленной цели необходимо решить \emph{следующие задачи}:
\begin{itemize}
\item проанализировать современные технологии и подходы к разработке графических движков;
\item разработать архитектуру кроссплатформенного графического движка;
\item реализовать основные компоненты графического конвейера;
\item разработать систему управления шейдерами, текстурами и геометрическими примитивами;
\item реализовать систему камеры и освещения сцены;
\item разработать формат файлов сцены и систему их загрузки;
\item провести тестирование и оптимизацию работы движка.
\end{itemize}

\emph{Структура и объем работы.} Отчет состоит из введения, 4 разделов основной части, заключения, списка использованных источников, 2 приложений. Текст выпускной квалификационной работы равен \formbytotal{lastpage}{страниц}{е}{ам}{ам}.

\emph{Во введении} сформулирована цель работы, поставлены задачи разработки, описана структура работы, приведено краткое содержание каждого из разделов.

\emph{В первом разделе} на стадии описания технической характеристики предметной области приводится анализ современных технологий и подходов к разработке графических движков.

\emph{Во втором разделе} на стадии технического задания приводятся требования к разрабатываемому движку.

\emph{В третьем разделе} на стадии технического проектирования представлены проектные решения для движка.

\emph{В четвертом разделе} приводится список классов и их методов, использованных при разработке движка, производится тестирование разработанного движка.

В заключении излагаются основные результаты работы, полученные в ходе разработки.

В приложении А представлен графический материал.
В приложении Б представлены фрагменты исходного кода. 
