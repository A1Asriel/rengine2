\abstract{РЕФЕРАТ}

Объем работы равен \formbytotal{lastpage}{страниц}{е}{ам}{ам}. Работа содержит \formbytotal{figurecnt}{иллюстраци}{ю}{и}{й}, \formbytotal{tablecnt}{таблиц}{у}{ы}{}, \arabic{bibcount} библиографических источников и \formbytotal{числоПлакатов}{лист}{}{а}{ов} графического материала. Количество приложений – 2. Графический материал представлен в приложении А. Фрагменты исходного кода представлены в приложении Б.

Перечень ключевых слов: графический движок, рендеринг, OpenGL, компьютерная графика, трёхмерная визуализация, шейдеры, текстуры, геометрия, камера, освещение, кроссплатформенность, SDL2, GLSL, GLM, C++, графический конвейер, прямая отрисовка, отложенная отрисовка, матрицы преобразований, буферы вершин, буферы индексов.

Объектом разработки является информационная система для визуализации трёхмерных сцен, реализованная на языке C++ с использованием современных графических технологий.

Целью выпускной квалификационной работы является разработка кроссплатформенного графического движка, обеспечивающего эффективный рендеринг трёхмерных сцен в реальном времени с использованием современных технологий компьютерной графики.

В процессе разработки были реализованы следующие основные компоненты: подсистема управления окнами и вводом, графический конвейер на базе OpenGL 3.3, система загрузки и управления шейдерами, подсистема работы с геометрией и текстурами, система камеры и освещения сцены, менеджер сцен и объектов.

Движок демонстрирует высокую производительность при рендеринге сложных трёхмерных сцен и может быть использован в различных областях, включая разработку интерактивных приложений, научную визуализацию и образовательные проекты.

\selectlanguage{english}
\abstract{ABSTRACT}
  
The volume of work is \formbytotal{lastpage}{page}{}{s}{s}. The work contains \formbytotal{figurecnt}{illustration}{}{s}{s}, \formbytotal{tablecnt}{table}{}{s}{s}, \arabic{bibcount} bibliographic sources and \formbytotal{числоПлакатов}{sheet}{}{s}{s} of graphic material. The number of applications is 2. The graphic material is presented in annex A. The layout of the site, including the connection of components, is presented in annex B.

List of keywords: graphics engine, rendering, OpenGL, computer graphics, 3D visualization, shaders, textures, geometry, camera, lighting, cross-platform, SDL2, GLSL, GLM, C++, graphics pipeline, deferred rendering, forward rendering, transformation matrices, vertex buffers, index buffers.

The object of development is a graphics engine for 3D scene visualization, implemented in C++ using modern graphics technologies.

The purpose of the final qualifying work is to develop a cross-platform graphics engine that provides efficient real-time rendering of 3D scenes using modern computer graphics technologies.

During the development, the following main components were implemented: window and input management subsystem, graphics pipeline based on OpenGL 3.3, shader loading and management system, geometry and texture handling subsystem, scene camera and lighting system, scene and object manager.

The engine demonstrates high performance in rendering complex 3D scenes and can be used in various fields, including interactive application development, scientific visualization, and educational projects.
\selectlanguage{russian}
