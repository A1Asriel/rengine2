\section{Техническое задание}
\subsection{Основание для разработки}

Основанием для разработки является задание на преддипломную практику по созданию графического движка для трёхмерной визуализации с использованием технологий OpenGL и SDL2.

\subsection{Цель и назначение разработки}

Основной целью преддипломной практики является разработка графического движка для визуализации трёхмерных объектов в реальном времени для ОАО <<Какао-Какао>>.

Посредством разработки данного движка планируется решить ряд технических задач, стоящих перед компанией. Цели разработки можно разделить на две основные группы: технические и практические.

Задачами данной разработки являются:
\begin{itemize}
\item реализация базовых компонентов графического конвейера на основе OpenGL;
\item создание кроссплатформенного решения с использованием SDL2;
\item разработка системы управления шейдерами и текстурами;
\item реализация системы камеры;
\item создание базовых геометрических примитивов;
\item разработка системы загрузки и отображения сцен.
\end{itemize}

\subsection{Требования к программному интерфейсу движка}

Графический движок должен предоставлять следующие программные интерфейсы:
\begin{itemize}
    \item интерфейс создания и управления окнами приложения;
    \item интерфейс управления шейдерами и программами для GPU;
    \item интерфейс работы с текстурами;
    \item интерфейс управления камерами;
    \item интерфейс работы с геометрическими примитивами;
    \item интерфейс загрузки и управления сценами.
\end{itemize}

Структура основных компонентов движка представлена на рисунке ~\ref{templ:image}.

\begin{figure}[ht]
% \includegraphics[width=1\linewidth]{templ}
\caption{Структура компонентов графического движка}
\label{templ:image}
\end{figure}
%\vspace{-\figureaboveskip} % двойной отступ не нужен (можно использовать, если раздел заканчивается картинкой)

\subsection{Моделирование вариантов использования}

Для разрабатываемого графического движка была построена модель, отражающая основные сценарии его использования в корпоративной среде.

Данная модель помогает выявить ключевые точки интеграции движка с существующими информационными системами предприятия, а также определить роли пользователей и взаимодействие с внешними системами. Для описания сценариев используется унифицированный язык визуального моделирования UML.

Диаграмма вариантов использования описывает функциональное назначение разрабатываемой системы. Она является исходным концептуальным представлением системы в процессе ее проектирования и разработки. Проектируемая система представляется в виде ряда прецедентов, предоставляемых системой актерам или сущностям, которые взаимодействуют с системой. Актером или действующим лицом является сущность, взаимодействующая с системой извне (например, человек, техническое устройство). Прецедент служит для описания набора действий, которые система предоставляет актеру.

На основании анализа предметной области в программе должны быть реализованы следующие прецеденты:
\begin{enumerate}
    \item отображение заранее созданных трёхмерных сцен с базовыми геометрическими примитивами (кубы, сферы);
    \item перемещение камеры по трёхмерной сцене с помощью клавиатуры и мыши;
    \item загрузка и применение текстур к объектам;
    \item поддержка вершинных и фрагментных шейдеров;
    \item загрузка сцен из внешних файлов (поддержка формата .csv);
    \item кроссплатформенная работа на операционных системах Windows и Linux;
    \item базовый рендеринг с поддержкой OpenGL 3.3.
\end{enumerate}

\subsection{Требования к программному обеспечению}

Для реализации разрабатываемого движка должны использоваться языки программирования C++ и GLSL, компилятор GCC и система сборки CMake.

Для запуска и работы с движком требуется использование компьютера под управлением операционной системы Windows 7 (или выше) или Linux.

\subsection{Требования к аппаратному обеспечению}

Клиентское оборудование должно иметь центральный процессор с частотой ядра не менее 1 ГГц и поддержкой SSE2, а также видеокарту с поддержкой OpenGL 3.3 или выше. Объём оперативной памяти -- 512 ГБ.

Доступ к сети Интернет не требуется.

\subsection{Требования к оформлению документации}

Разработка программной документации и программного изделия должна производиться согласно ГОСТ 19.102-77 и ГОСТ 34.601-90. Единая система программной документации.
