\appendix{Представление графического материала}

Графический материал, выполненный на отдельных листах,
изображен на рисунках А.1--А.\arabic{числоПлакатов}.
\setcounter{числоПлакатов}{0}

\renewcommand{\thefigure}{А.\arabic{figure}} % шаблон номера для плакатов

\begin{landscape}

\begin{плакат}
    \includegraphics[width=0.82\linewidth]{p1}
    \заголовок{Сведения о ВКРБ}
    \label{pl1:image}
\end{плакат}

\begin{плакат}
    \includegraphics[width=0.82\linewidth]{p2}
    \заголовок{Цель и задачи разработки}
    \label{pl2:image}
\end{плакат}

\begin{плакат}
    \includegraphics[width=0.82\linewidth]{p3}
    \заголовок{Диаграмма компонентов}
    \label{pl3:image}
\end{плакат}

\begin{плакат}
    \includegraphics[width=0.82\linewidth]{p4}
    \заголовок{Диаграмма классов}
    \label{pl4:image}
\end{плакат}

\begin{плакат}
    \includegraphics[width=0.82\linewidth]{p5.png}
    \заголовок{Изображение основной сцены}
    \label{pl5:image}
\end{плакат}

\begin{плакат}
    \includegraphics[width=0.82\linewidth]{p6.png}
    \заголовок{Тестирование оптимизации}
    \label{pl6:image}
\end{плакат}

\begin{плакат}
    \includegraphics[width=0.82\linewidth]{p7.png}
    \заголовок{Освещение по модели Блинна-Фонга}
    \label{pl7:image}
\end{плакат}

\begin{плакат}
    \includegraphics[width=0.82\linewidth]{p8.png}
    \заголовок{Демонстрация гибкости пользовательского ввода}
    \label{pl8:image}
\end{плакат}

\begin{плакат}
    \includegraphics[width=0.82\linewidth]{p9}
    \заголовок{Заключение}
    \label{pl9:image}
\end{плакат}

\end{landscape}
