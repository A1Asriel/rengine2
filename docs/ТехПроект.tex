\section{Технический проект}
\subsection{Общая характеристика архитектуры решения}

Графический движок представляет собой кроссплатформенное приложение для визуализации трёхмерных сцен, построенное на следующих ключевых компонентах:
\begin{itemize}
    \item подсистема управления окнами (SDL2);
    \item графический конвейер (OpenGL 3.3);
    \item подсистема загрузки ресурсов (текстуры, шейдеры);
    \item менеджер сцен и объектов.
\end{itemize}

Движок спроектирован как модульная система, позволяющая расширять функционал без изменения ядра.

\subsection{Обоснование выбора технологий}

\subsubsection{Язык программирования C++}

В качестве основного языка разработки выбран C++ -- высокопроизводительный компилируемый язык, обеспечивающий прямой доступ к графическим API через нативные биндинги. Его кроссплатформенная природа позволяет компилировать движок под различные операционные системы без изменения исходного кода. Современные стандарты C++ предоставляют необходимые средства для эффективной работы с памятью и многопоточностью, что критично для графических приложений.

\subsubsection{SDL2}

Для взаимодействия с оконной системой используется библиотека SDL2 (Simple DirectMedia Layer), которая абстрагирует платформо-зависимые особенности создания окон, обработки ввода и работы с аудио. SDL2 была выбрана благодаря своей стабильности, широкой поддержке различных платформ (Windows, Linux, macOS) и минимальным затратам времени на адаптацию. Библиотека предоставляет простой API для инициализации графического контекста OpenGL и обработки пользовательского ввода.

\subsubsection{OpenGL}

Графический конвейер реализован на основе OpenGL 3.3 Core Profile - кроссплатформенного графического API, поддерживаемого большинством современных видеокарт. Выбор версии 3.3 обусловлен балансом между функциональностью и совместимостью: этот стандарт предоставляет современный конвейер рендеринга с шейдерной моделью, но при этом не требует новейшей аппаратуры. OpenGL был предпочтен Vulkan и Direct3D ввиду своей универсальности и меньшего порога входа.

\subsubsection{GLM (OpenGL Mathematics)}

Библиотека GLM предоставляет математические функции и типы данных, специфичные для компьютерной графики:
\begin{itemize}
    \item векторные и матричные операции (vec3, mat4);
    \item трансформации (перемещение, вращение, масштабирование);
    \item пространственных преобразований (нормализация, проекции).
\end{itemize}

GLM применяется в движке для упрощения таких действий, как расчёт матриц модели-вида-проекции, преобразование координат в шейдерах и управление камерой и перспективой.

Библиотека была выбрана благодаря полной совместимости с OpenGL, оптимизированным SIMD-операциям и удобному синтаксису, аналогичному GLSL.

\subsection{Архитектурные компоненты системы}

Диаграмма компонентов (рис. \ref{comp:image}) отражает физическую и логическую структуру графического движка. Архитектура системы построена по принципу разделения ответственности, где каждый компонент инкапсулирует строго определённую функциональность.

\subsubsection{Состав компонентов}
Основные модули системы включают:

\begin{enumerate}
    \item RE\_Window -- компонент управления оконным контекстом, реализующий абстракцию над библиотекой SDL2. Обеспечивает:
    \begin{itemize}[itemindent=\parindent,leftmargin=\parindent]
        \item создание и настройку графического окна;
        \item инициализацию контекста OpenGL;
        \item обработку системных событий и пользовательского ввода.
    \end{itemize}

    \item Scene и SceneLoader -- подсистема работы со сценой:
    \begin{itemize}[itemindent=\parindent,leftmargin=\parindent]
        \item SceneLoader загружает иерархию объектов из CSV-файлов;
        \item Scene хранит коллекцию объектов и управляет их состоянием.
    \end{itemize}

    \item Графический конвейер (неявный компонент):
    \begin{itemize}[itemindent=\parindent,leftmargin=\parindent]
        \item Shader -- управление шейдерными программами (вершинный/фрагментный);
        \item Texture -- загрузка, генерация и привязка текстурных объектов;
        \item Mesh/СubeMesh/SphereMesh -- хранение геометрических данных (VBO/VAO/EBO).
    \end{itemize}

    \item Camera -- компонент управления видами, реализующий:
    \begin{itemize}[itemindent=\parindent,leftmargin=\parindent]
        \item расчёт матриц вида и проекции;
        \item преобразование координат;
        \item управление параметрами отображения.
    \end{itemize}
\end{enumerate}

\subsubsection{Взаимодействие компонентов}
Последовательность работы системы (рис. \ref{comp:image}) реализуется следующим образом:

\begin{enumerate}
\item Инициализация:
   \begin{itemize}[itemindent=\parindent,leftmargin=\parindent]
       \item RE\_Window создаёт графический контекст
       \item Shader компилирует шейдерные программы
       \item SceneLoader загружает начальную сцену
   \end{itemize}

\item Главный цикл рендеринга:
   \begin{itemize}[itemindent=\parindent,leftmargin=\parindent]
       \item Обновление матриц вида в Camera.
       \item Передача uniform-переменных в шейдеры.
       \item Отрисовка объектов сцены с учётом их:
       \begin{itemize}[itemindent=\parindent,leftmargin=\parindent]
           \item геометрии (Mesh);
           \item текстуры (Texture);
           \item шейдеров (Shader).
       \end{itemize}
       \item Вывод результата через RE\_Window.
   \end{itemize}
\end{enumerate}

\begin{figure}[ht]
\centering
% \includegraphics[width=0.9\textwidth]{engine_components}
\caption{Диаграмма компонентов движка с последовательностью взаимодействия}
\label{comp:image}
\end{figure}

\subsection{Основные сущности системы}

В таблице \ref{tab:entities} приведено описание основных сущностей графического движка.

\begin{xltabular}{\textwidth}{|c|X|X|}
\caption{Сущности графического движка\label{tab:entities}}\\ \hline
\centrow Сущность & \centrow Тип & \centrow Описание \\ \hline
\endfirsthead
\continuecaption{Продолжение таблицы \ref{tab:entities}}
\centrow Сущность & \centrow Тип & \centrow Описание \\ \hline
\finishhead

Camera & Класс & Камера в сцене \\ \hline
RE\_Window & Класс & Окно приложения \\ \hline
Texture & Класс & Хранит данные текстур и методы работы с ними \\ \hline
Shader & Класс & Управление шейдерными программами (вершинный/фрагментный) \\ \hline
Mesh & Абстрактный класс & Базовый класс для геометрических примитивов \\ \hline
CubeMesh & Наследник Mesh & Реализация кубической геометрии \\ \hline
SphereMesh & Наследник Mesh & Реализация сферической геометрии \\ \hline
Scene & Класс & Контейнер для объектов и состояния сцены \\ \hline
SceneLoader & Статический класс & Методы для загрузки и сохранения сцены
\end{xltabular}

\subsection{Формат файла сцены}

Движок использует CSV-формат для хранения сцен. Структура файла описана в таблице \ref{tab:scene_spec}.

\begin{xltabular}{\textwidth}{|l|l|X|}
\caption{Спецификация формата scene.csv\label{tab:scene_spec}}\\ \hline
\centrow Поле & \centrow Тип & \centrow Описание \\ \hline
\endfirsthead
\continuecaption{Продолжение таблицы \ref{tab:scene_spec}}
\centrow Поле & \centrow Тип & \centrow Описание \\ \hline 
\finishhead

\multicolumn{3}{|c|}{Объекты сцены} \\ \hline
mesh & string & Тип объекта (cube/sphere) \\ \hline
pos\_x, pos\_y, pos\_z & float & Положение объекта в пространстве \\ \hline
rot\_x, rot\_y, rot\_z & float & Углы поворота объекта (в градусах) \\ \hline
scale\_x, scale\_y, scale\_z & float & Масштабирование по осям \\ \hline
distort & bool & Флаг искажения объекта \\ \hline
texture & string & Путь к файлу текстуры \\ \hline

\multicolumn{3}{|c|}{Параметры камеры} \\ \hline
camera\_x, camera\_y, camera\_z & float & Позиция камеры \\ \hline
rotate\_x, rotate\_y, rotate\_z & float & Углы поворота камеры (в градусах) \\ \hline
fov & float & Угол обзора (в градусах)
\end{xltabular}

Пример содержимого файла:
\begin{lstlisting}[language=,caption=Пример scene.csv]
/// Camera, x, y, z, rx, ry, rz, FOV ///
camera,0,0,2,0,180,0,60

/// Mesh type (cube/sphere), x, y, z, rx, ry, rz, sx, sy, sz, distort, texture path (optional) ///
cube,0.0,0.5,-2.0,0,1.5,0,1.0,1.0,1.0,false,./textures/wood.jpg
sphere,1.0,-0.5,-3.0,0,0,0,1.0,1.0,1.0,false,./textures/metal.jpg
\end{lstlisting}
