\section*{ЗАКЛЮЧЕНИЕ}
\addcontentsline{toc}{section}{ЗАКЛЮЧЕНИЕ}

В ходе выполнения выпускной квалификационной работы была разработана кроссплатформенная система визуализации трёхмерной графики, обеспечивающая эффективный рендеринг сложных сцен в реальном времени. Разработанное программное решение демонстрирует высокую производительность и расширяемость, что позволяет применять его в различных областях, включая разработку интерактивных приложений, научную визуализацию и образовательные проекты.

Основные результаты работы:

\begin{enumerate}
    \item Проведён всесторонний анализ современных технологий и подходов в области компьютерной графики:
    \begin{itemize}
        \item исследованы современные графические API (OpenGL, Vulkan, DirectX);
        \item проанализированы существующие игровые движки и фреймворки;
        \item выбран оптимальный стек технологий: C++17, OpenGL 3.3, SDL2, GLM, GLSL.
    \end{itemize}
    
    \item Разработана архитектура кроссплатформенного графического движка:
    \begin{itemize}
        \item определены ключевые модули системы: рендеринг, управление ресурсами, обработка ввода;
        \item разработана система управления жизненным циклом приложения;
        \item спроектированы интерфейсы для расширения функциональности.
    \end{itemize}
    
    \item Реализованы основные компоненты графического конвейера:
    \begin{itemize}
        \item разработана система управления шейдерами с поддержкой загрузки пользовательского кода;
        \item реализована подсистема загрузки и управления текстурами;
        \item создана система работы с геометрическими примитивами и буферами вершин.
    \end{itemize}
    
    \item Разработана система работы с камерой и освещением сцены:
    \begin{itemize}
        \item реализован динамический объект камеры, имеющий возможность изменения позиции и поля обзора;
        \item добавлена поддержка источников света (направленные и точечные), рассчитываемых по модели Блинна-Фонга.
    \end{itemize}
    
    \item Создана система загрузки и управления сценами:
    \begin{itemize}
        \item реализован загрузчик графических файлов формата BMP;
        \item проведена оптимизация загрузки ресурсов с помощью кеширования;
        \item добавлена поддержка различных типов объектов (модели, источники света, камера).
    \end{itemize}
    
    \item Реализована система обработки пользовательского ввода:
    \begin{itemize}
        \item добавлена поддержка клавиатуры и мыши;
        \item реализована система настройки управления;
        \item обеспечена плавность управления камерой.
    \end{itemize}
    
    \item Проведены тестирование и оптимизация работы движка:
    \begin{itemize}
        \item реализованы модульные тесты для всех компонентов;
        \item проведено профилирование производительности;
        \item выполнена оптимизация рендеринга (отсечение объектов за пределами кадра).
    \end{itemize}
\end{enumerate}

Все требования, объявленные в техническом задании, были полностью реализованы, все задачи, поставленные в начале разработки проекта, были также решены. Разработанный информационная система демонстрирует стабильную работу на различных аппаратных конфигурациях и операционных системах. Достигнута частота кадров, достаточная для интерактивной работы со сложными трёхмерными сценами.

Готовый рабочий проект представлен в виде программной библиотеки на C++. Исходный код находится в публичном доступе, поскольку опубликован в сети Интернет.
