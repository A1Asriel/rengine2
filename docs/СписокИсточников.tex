\addcontentsline{toc}{section}{СПИСОК ИСПОЛЬЗОВАННЫХ ИСТОЧНИКОВ}

\begin{thebibliography}{9}
    \bibitem{opengl-es} Будирижанто, П. OpenGL ES 3.0. Руководство разработчика / П. Будирижанто, Г. Дэн. – Москва~: ДМК Пресс, 2015. – 448 с. – ISBN 978-5-97060-256-0. – Текст~: непосредственный.
    \bibitem{cpp} Доусон, М. Изучаем C++ через программирование игр / М. Доусон. – Санкт-Петербург~: Питер, 2022. – 352 с. – ISBN 978-5-4461-1791-8. – Текст~: непосредственный.
    \bibitem{lafore} Лафоре, Р. Объектно-ориентированное программирование в С++. Классика Computer Science / Р. Лафоре. – Санкт-Петербург~: Питер, 2022. – 928 с. – ISBN 978-5-4461-0927-2. – Текст~: непосредственный.
    \bibitem{shildt} Шилдт Герберт. C++. Базовый курс / Г. Шилдт. – Москва~: Диалектика-Вильямс, 2018. – 624 с. – ISBN 978-5-907114-15-9. – Текст~: непосредственный.
    \bibitem{deitel} Дейтел П. C++20 для программистов / П. Дейтел, Х. Дейтел. – Санкт-Петербург~: Питер, 2024. – 1056 с. – ISBN 978-5-4461-2359-9. – Текст~: непосредственный.
    \bibitem{macconnell} Макконнелл, Д. Анализ алгоритмов. Активный обучающий подход. 3-е дополненное издание / Д. Макконнелл. – Москва~: Техносфера, 2023. – 416 с. – ISBN 978-5-94836-216-8. – Текст~: непосредственный.
    \bibitem{kovalyov} Ковалев, М. М. Дискретная оптимизация: Целочисленное программирование / М. М. Ковалев. – Москва~: Ленанд, 2023. – 192 с. – ISBN 978-5-9710-7853-1. – Текст~: непосредственный.
    \bibitem{mitchell} Митчелл, Ш. Разработка игр на SDL / Ш. Митчелл. – Москва~: Пакет, 2013. – 256 с. – ISBN 978-1-84969-682-1. – Текст~: непосредственный.
    \bibitem{guidukov} Гайдуков, С. А. OpenGL. Профессиональное программирование трехмерной графики на C++ / С. А. Гайдуков. – Санкт-Петербург~: БХВ-Петербург, 2011. – ISBN 978-5-94157-363-4. – Текст~: электронный.
    \bibitem{discrete} Авдошин, С. М. Дискретная математика. Алгоритмы: теория и практика / С. М. Авдошин, А. А. Набебин. – Москва~: ДМК Пресс, 2019. – 282 с. – ISBN 978-5-97060-688-9. – Текст~: непосредственный.
    \bibitem{data} Манцнер, Т. Визуализация данных. Полный и исчерпывающий курс для начинающих / Т. Манцнер. – Москва~: Эксмо, 2023. – 464 с. – ISBN 978-5-04-106797-7. – Текст~: непосредственный.
    \bibitem{knuth} Кнут, Д. Э. Искусство программирования. Том 3. Сортировка и поиск / Д. Э. Кнут. – Москва~: Диалектика Вильямс, 2019. – 832 с. – ISBN 978-5-907144-41-5. – Текст~: непосредственный.
    \bibitem{grok} Бхаргава, А. Грокаем алгоритмы. 2-е изд. / А. Бхаргава. – Санкт-Петербург~: Питер, 2025. – 352 с. – ISBN 978-5-4461-4172-2. – Текст~: непосредственный.
    \bibitem{c} Ритчи, Д. М. Язык программирования C | Д. М. Ритчи, Б. У. Керниган – Москва~: Вильямс, 2019. – 288 с. – ISBN 978-5-907144-14-9. – Текст~: непосредственный.
    \bibitem{test} Кейнер, К. Тестирование программного обеспечения: контекстно ориентированный подход / К. Кейнер, Дж. Бах. – Санкт-Петербург~: Питер, 2025. – 352 с. – ISBN 978-5-4461-2165-6. – Текст~: непосредственный.
    \bibitem{csharp} Тюкачев, Н. А. C\#. Программирование 2D и 3D векторной графики. Учебное пособие для вузов, 5-е изд., стер. / Н. А. Тюкачев, В. Г. Хлебостроев. – Санкт-Петербург~: Лань, 2024. – 320 с. – ISBN 978-5-507-47565-0. – Текст~: непосредственный.
    \bibitem{game} Андрианова, Н. А. Как создаются игры. Основы разработки для начинающих игроделов / Н. А. Андрианова. – Москва~: Бомбора, 2023. – 336 с. – ISBN 978-5-04-120353-5. – Текст~: непосредственный.
    \bibitem{test2} Элфрид, Д. Автоматизированное тестирование программного обеспечения / Д. Элфрид, Р. Джефф, П. Джон. – Москва~: Лори, 2023. – 592 с. – ISBN 978-5-85582-409-4. – Текст~: непосредственный.
    \bibitem{req} Вигерс, К. И. Разработка требований к программному обеспечению. 3-е изд., дополненное / К. И. Вигерс, Дж. Битти. – Санкт-Петербург~: БХВ, 2024. – 736 с. – ISBN 978-5-9909805-3-2. – Текст~: непосредственный.
    \bibitem{wolf} Вольф, Д. OpenGL 4. Язык шейдеров. Книга рецептов / Д. Вольф. – Москва~: ДМК Пресс, 2017. – ISBN 978-5-97060-255-3. – Текст~: электронный.
\end{thebibliography}
